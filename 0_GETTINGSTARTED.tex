\documentclass[12pt]{article}
\usepackage{hyperref}


\title{This is a template document.}

\begin{document}
\maketitle

This document (0\_GETTINGSTARTED.tex) should introduce a new user of the standardized File System Structure (FSS) to the most important aspects of your project. This should allow him/her to quickly reproduce the main results. You can include figures and tables where-ever needed. \\ 

You should link relevant directories and/or files from this document by defining relative hyperlinks (i.e., relative to the root of the FSS) to specific sub-directories or files. For example:
\begin{itemize}
  \item \href{file:./Processing}{\underline{./Processing}}
  \item You can find the code \href{file:./Processing}{[\underline{HERE}]}
  \item \href{file:./Data/0\_README.md}{\underline{./Data/0\_README.md}}
\end{itemize}


You can copy this text file to your favorite text processor (e.g., Microsoft Word, Mac Pages, LaTex). The final document should be converted to html (0\_GETTINGSTARTED.html) since that file will be used by the FSS Navigator (Navigate.py, Navigate.exe, or the Unix script Navigate) \\

A possible structure for this document is:
\\ \\
Aim:
\\ \\
Data:
\\ \\ 
Software:
\\ \\
Main results:
\\ \\
Conclusion:
\\
\end{document}
